\documentclass{article}\usepackage{noweb}\pagestyle{noweb}\noweboptions{}\begin{document}\nwfilename{1.nw}\nwbegindocs{0}\section*{Сортировка вставками}% ===> this file was generated automatically by noweave --- better not edit it

Рассмотрим реализацию алгоритма сортировки одномерного массива методом вставок.

\nwenddocs{}\nwbegincode{1}\moddef{insertionSort.py}\endmoddef\nwstartdeflinemarkup\nwenddeflinemarkup
def insertion_sort(arr):
    \LA{}Основной цикл\RA{}
\nwendcode{}\nwbegindocs{2}\nwdocspar

Создаём цикл, проходящий по массиву arr, начиная со второго элемента, так как считается, что первый элемент уже отсортирован

\nwenddocs{}\nwbegincode{3}\moddef{Основной цикл}\endmoddef\nwstartdeflinemarkup\nwenddeflinemarkup
for i in range(1, len(arr)):
    \LA{}Сохранение текущего элемента\RA{}
    \LA{}Сохранение индекса предыдущего числа\RA{}
    \LA{}Поиск позиции для вставки\RA{}
    \LA{}Вставка элемента\RA{}
\nwendcode{}\nwbegindocs{4}\nwdocspar

Сохраним число из массива с индексом \texttt{i} в переменную \texttt{item}.

\nwenddocs{}\nwbegincode{5}\moddef{Сохранение текущего элемента}\endmoddef\nwstartdeflinemarkup\nwenddeflinemarkup
item = arr[i]
\nwendcode{}\nwbegindocs{6}\nwdocspar

Сохраним индекса предыдущего числа в переменную \texttt{j}

\nwenddocs{}\nwbegincode{7}\moddef{Сохранение индекса предыдущего числа}\endmoddef\nwstartdeflinemarkup\nwenddeflinemarkup
j = i - 1
\nwendcode{}\nwbegindocs{8}\nwdocspar

С помощью цикла \texttt{while} находим такую позицию в массиве, при которой число в переменной \texttt{item} будет меньше чем \texttt{arr[j]}.

\nwenddocs{}\nwbegincode{9}\moddef{Поиск позиции для вставки}\endmoddef\nwstartdeflinemarkup\nwenddeflinemarkup
while j >= 0 and arr[j] > item:
    \LA{}Сдвиг элементов в массиве\RA{}
    j -= 1
\nwendcode{}\nwbegindocs{10}\nwdocspar

При каждой итерации цикла \texttt{while} сдвигаем элемент массива с индексом \texttt{j} на одну позиции вперёд.

\nwenddocs{}\nwbegincode{11}\moddef{Сдвиг элементов в массиве}\endmoddef\nwstartdeflinemarkup\nwenddeflinemarkup
arr[j + 1] = arr[j]
\nwendcode{}\nwbegindocs{12}\nwdocspar

Производим вставку элемента в массив по индексу \texttt{j + 1}

\nwenddocs{}\nwbegincode{13}\moddef{Вставка элемента}\endmoddef\nwstartdeflinemarkup\nwenddeflinemarkup
arr[j + 1] = item
\nwendcode{}\nwbegindocs{14}\nwdocspar

\footnote{Разработал Ворожейкин Д.А. ИКБО-01-19}
\nwenddocs{}\end{document}

